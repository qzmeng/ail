

\documentclass{beamer}
\begin{document}

\setlength{\parskip}{6pt}

\frame{
  \frametitle{What is Text Classification}
  
  Text classification is automatically classifying a document into two or more 
  \emph{predefined} categories.
  
  Examples of documents and their possible categories:
  Emails: spam or not spam
  Newswire articles: business, politics or sports
  Movie reviews: liked it or hated it
  
}
  
\frame{
  \frametitle{Supervised learning}
  
  What knowledge do we need to complete the task?
  
  Classifying newswires into politics or sports:
  Read the newswires, find the characteristics which determine whether
  they are politics or sports, then make a judgement.
  
  But what if you've never read a politics article before?  How would you
  know what to look for?
  
  \pause

  In \emph{supervised learning}, the machine learns based on examples and their
  {\bf already known} classifications ({\it training set})
  
  Based on the knowledge learnt from the examples, we can classify 
  {\bf unknown} examples ({\it test set})
}

\frame{
  \frametitle{Today's problem}  
  
  Lady Gaga vs. The Clash
  
  
  
  Input to this problem: a bunch of text files containing song lyrics.
  
}

\frame{
  \frametitle{Preprocessing the input}
  
Decompose (tokenise) each input file into a `bag of words'.
  
 Bag of words: a list of words with corresponding frequency of that word
occurring in the document (a weighted vector of words)
  
  
  \begin{tabular}{l|lllll}
  & $word_1$ & $word_2$ & $word_3$ & $word_4$ & $word_5$ \\
 \hline
  $doc_1$ & & & & & \\
  $doc_2$ & & & & & \\
  $doc_3$ & & & & & \\

  \end{tabular}
  


}

\frame{
  \frametitle{Train the classifier}
  
  Multinomial Naive Bayes: a Generative learning algorithm
  
  Generative learning algorithms create a model based on the training data,
  and apply that model on the test data.
  
  Two categories: ${gaga, clash}$
  
  We will build a model for each category.
  
  Classification task: find the \emph{most likely} category that the 
  document belongs to.
  
}

\frame{
  \frametitle{Demo}
  
  
}

\frame{
  \frametitle{How does it work?}
  
  
}

\frame{
  \frametitle{Bayes' Theorem}

  $$ P(A|B) = \frac{P(B|A)P(A)}{P(B)} $$
  
  Where:

  $A$ and $B$ are events

  $P(B) \neq 0$ and is the probability of the data

  $P(A|B)$: probability that $A$ occurs given $B$

  $P(B|A)$: probability that $B$ occurs given $A$
}

\frame{
  \frametitle{Bayes' Theorem in Text Classification}

  Objective: find the likelihood that document $d$ belongs in class
  $c$

  $$P(c|d) = \frac{P(d|c)P(c)}{P(d)}$$ 


  Try each of the classes in turn i.e.   $C=\{gaga,clash\}$

  The most likely class is the one which returns the highest value of
  $P(c|d)$ which is defined as $C_{MAP}$ (maximum a posteriori)
  $C_{MAP} = \stackrel{\displaystyle \mathrm{arg\  max}}{\scriptstyle c \in C} P(c|d)$

  \pause

  {\it (Apply Bayes' theorem again)}

  $C_{MAP} = \stackrel{\displaystyle \mathrm{arg\  max}}{\scriptstyle c \in C} \frac{P(d|c)P(c)}{P(d)}$

  $C_{MAP} = \stackrel{\displaystyle \mathrm{arg\  max}}{\scriptstyle c \in C} P(d|c)P(c)$
}

\frame{
  \frametitle{Training}

  Recall that our document ($d$) is just a `bag of words'.  We can
  represent the words as the product of all of the word probabilties:

  $$P(c|d) \propto P(c) \prod_{1 \leq k \leq n_d} P(t_k|c)$$


  Prior probability of the class:
  
  $P(C)=\frac{N_c}{N}$

  Where $N_c$ is the number of docs in the class, and $N$ is the total
  number of docs.  


  Relative frequency of term $t$ occurring in a class $c$:

  $$P(t|c) = \frac{T_{ct}}{\sum_{t \in V} T_{ct}}$$

  where
  $T_{ct}$ is the number of occurrences of $t$ in training documents
  from class $c$. ($V$ is the set of all terms)

}


\frame{
  \frametitle{The problem of zero}

  Since we need to iterate through all terms in our vocabulary for all classes, if a document in a class doesn't include a term existing in
  the other classes, $T_{ct}=0$, which makes $P(c|d)=0$ when
  included in the product of all terms.  


  For example, $gaga$ documents include the word `mascara' which
  appears in no $clash$ documents, so $P(mascara|clash)=0$.

  We fix this by applying {\it add-one smoothing}:
 
  $$P(t|c) = \frac{T_{ct}}{\sum_{t \in V} T_{ct}}$$ 
  
  becomes

  $$P(t|c)=\frac{T_{ct}+1}{\sum_{t \in V}  ( T_{ct} + 1 )  } = 
    \frac{T_{ct}+1} {(\sum_{t \in V} T_{ct}) + |V|)}$$ 
  
   where $|V|$ is the total number of terms in the vocabulary.
   
}


\frame{
  \frametitle{Example: training}

  Training set:

  \begin{tabular}{l|r|r|r|r|r|r}
    
         &$N_c$ &  animal & game & love & london  & $T_{ct}$ \\ \hline
    gaga & 2  & 66 & 1 & 21 &  0 & {\it 88} \\
    clash& 2  &  0 & 4 &  0 & 14 & {\it 18} \\

  \end{tabular}

  Class prior: $P(c)=\frac{N_c}{N} = \frac{2}{4}$ for both classes so we can dismiss it.

  Term likelihood: 

  \begin{tabular}{l|r|r|r|r}
    
           & animal & game & love & london  \\ \hline
    gaga   &  \only<1>{\bf $P(animal|gaga)$}\only<2-4>{0.345361} & \only<4>{0.010309}  & \only<4>{0.113402}  & \only<4>{0.005155}  \\
    clash  &   \only<2>{\bf $P(animal|clash)$}\only<3-4>{0.008065} & \only<4>{0.040323}  & \only<4>{0.008605}  &  \only<4>{0.120968}   \\

  \end{tabular}

  $P(t|c)=\frac{T_{ct}+1} {(\sum_{t \in V} T_{ct}) + |V|)}$

  $P(animal|gaga)=\frac{66+1}{88+106} = 0.345361$

  \only<1>{$ $}\only<2-4>{$P(animal|clash)=\frac{0+1}{18+106} = 0.008065$}

}

\frame{
  \frametitle{Example: classifying}

  Our trained classifier:

  \begin{tabular}{l|r|r|r|r}
    
           & animal & game & love & london  \\ \hline
    gaga   &  0.345361 & 0.010309 & 0.113402  & 0.005155  \\
    clash  &  0.008065 & 0.040323 & 0.008605  & 0.120968  \\

  \end{tabular}

  Test document:
  
  \begin{tabular}{l|r|r|r|r}
    
           & animal & game & love & london  \\ \hline
    ??? &   & 1 & 2 & 1  \\

  \end{tabular}


  \pause

  Multiply the test document by the training set:

  \begin{tabular}{l|r|r|r|r}
    
           & animal & game & love & london  \\ \hline
    gaga   &   & 0.010309 & 0.226804  & 0.005155  \\
    clash  &   & 0.040323 & 0.016129  & 0.120968  \\

  \end{tabular}

  \pause

  This gives:

 $P(gaga) = 0.010309 * 0.226804 * 0.005155 = 6.02625\text{e}^{-6}$

 $P(clash)= 0.040323 * 0.016129 * 0.120968 = 3.93365\text{e}^{-5}$ \pause  \fbox{Wins}


}

\frame{
  \frametitle{References}


Christopher D. Manning, Prabhakar Raghavan and Hinrich Schütze. (2008) {\it
  Introduction to Information Retrieval}, Cambridge University
Press.  {\tt https://nlp.stanford.edu/IR-book/}

}

\end{document}

